\begin{abstract}
Phenomena such as electromagnetic forces scale with linear dimensions in ways that make small spacecraft qualitatively different from larger ones.  Specifically, small scale may be capable of a new kind of mission architecture, inspecting and servicing larger spacecraft in close proximity without mechanical contact. Nearly all current technologies for applying force and torque between two spacecraft share a disadvantage: they require either propellant or mechanical contact. By using the force between a magnetic field and the electric currents it induces in a conductive target, a new technology known as an induction coupler exploits eddy-current effects to control the relative position and orientation between a chaser spacecraft and a target without mechanical contact. The induction coupler does not rely on familiar magnetic attraction and repulsion, which would require ferromagnetic materials on the target.  In contrast, and induction coupler is broadly applicable for even uncooperative targets, as long as the target includes conductive materials, as is generally the case for spacecraft.  This paper presents an overview of an induction coupler system, outlines design requirements through a case study of an inspection mission relevant to the International Space Station, and establishes the feasibility of flight applications through a description of ongoing experimental work.
\end{abstract}
